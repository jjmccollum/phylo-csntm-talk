\documentclass[tikz]{standalone}
\usepackage{../sty/phylogenetic-diagrams} % Include the style file containing the code defining the structure and style of the document

\begin{document}
\begin{tikzpicture}[node distance=2.5cm]
	\node[witReading] (costs_rows) at (-6.5,2) {%
		\begin{tabular}{r}
			{\color{purple}\textgreek{ιησου}}\\
			{\color{red}\textgreek{χριστου ιησου}}\\
			{\color{blue}\textgreek{ιησου χριστου}}\\
			{\color{darkgreen}\textgreek{χριστου}}
		\end{tabular}
	};
	\node[witReading, rotate=45] (costs_col_1) at (-4.75,3.25) {{\color{purple}\textgreek{ιησου}}};
	\node[witReading, rotate=45] (costs_col_2) at (-3.75,3.625) {{\color{red}\textgreek{χριστου ιησου}}};
	\node[witReading, rotate=45] (costs_col_3) at (-3.125,3.625) {{\color{blue}\textgreek{ιησου χριστου}}};
	\node[witReading, rotate=45] (costs_col_4) at (-2.75,3.375) {{\color{darkgreen}\textgreek{χριστου}}};
	\node[witReading] (costs) at (-4,2) {%
		\begin{tabular}{|c|c|c|c|}
			\hline
			$0$ & $3$ & $3$ & $5$\\
			\hline
			$10$ & $0$ & $5$ & $2$\\
			\hline
			$2$ & $5$ & $0$ & $10$\\
			\hline
			$5$ & $3$ & $3$ & $0$\\
			\hline
		\end{tabular}
	};
\end{tikzpicture}
\end{document}